
\chapter*{Abstract}
\addcontentsline{toc}{chapter}{Abstract}

This dissertation utilized two areas of computational simulations for organic molecular structure prediction. First, we investigate the possible polymorphism of both barbituric acid dihydrate (BTADH) and violuric acid monohydrate (VAMH) using density functional theory. Crystalline structures that exhibit polymorphism have an increased level of uncertainty in predicting their chemical and physical properties. BTADH was previously thought to undergo a subtle phase transition from \textit{P}2\textsubscript{1}/\textit{n} to planar \textit{Pnma} at 217 K. We conclude that barbituric acid dihydrate exhibits crystalline disorder in which the molecules are crystallized in a non-merohedrally twinned \textit{P}2\textsubscript{1}/\textit{n} configuration below 217 K. At temperatures above 217 K, however rather than a phase transition to \textit{Pnma}, the molecules have enough thermal energy to oscillate between the two twinned configurations. A similar planar configuration was reported for violuric acid monohydrate, which lead us to investigate the crystalline configuration. In modeling violuric acid monohydrate we redetermined the space group from the planar \textit{Cmc}2\textsubscript{1} configuration to a non-planar  \textit{Pca}2\textsubscript{1}. Both VAMH and BTADH exhibited molecule oscillations between twinned configurations that led to the incorrect determination of the planar space groups. 
Moving away from first principle simulations, we began to explore the realm of molecular dynamics. As opposed to density functional theory, MD simulations make approximations that significantly lower the computational cost. Due to this, we were able to investigate the bulk morphologies of larger molecule systems, such as macromolecules and polymers. Investigating these systems has been proven difficult due to the lack of molecular information, such as bond angles and dihedral parameters. In this study, we implemented and validated the use of machine-learned forcefield parameters for macromolecules and complex polymer systems. We created an end-to-end workflow that allowed us to predict the morphologies of these complex systems beginning with a simple SMILES string of the molecules. We then used this validated workflow to predict the morphology of 13 donor-acceptor co-polymers with photovoltaic properties. Lastly, we built a function to determine the rigidity of the 13 co-polymers and compared the simulation results to published experimental results in efforts to investigate the physical properties of these polymers from the predicted morphologies. 

%%% Local Variables: 
%%% mode: latex
%%% TeX-master: "BSUmain"
%%% End: 
