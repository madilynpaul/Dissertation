\chapter{Conclusions}
\label{chap:Conclusions}

Over the course of this dissertation we have tied together two very different, but theoretically similar projects. Beginning with attempting to model the polymorphic phase transition of barbituric acid dihydrate, we discovered that what was reported to be a very subtle phase transition at 217 K was actually the presence of crystalline disorder. We aimed to model this phase transition via density functional theory (DFT) due to the present of hydrate molecules in the crystal structure, but when two equivalent energy well were observed in the potential energy scan of the structure at temperatures above 217 K it became clear that this phase transition might not exist. This paired with the widening of the peaks in the terahertz spectra, lead us to conclude that there was twinning within the crystalline structure. Further investigation into this revealed that above 217 K the molecules had enough energy to oscillate between the two energetically favorable configurations. When this structure was analyzed using x-ray crystallography the oscillating molecule's positions were averaged out into a perfectly planar configuration, leading the investigators to report that the molecules were confined to a \textit{Pnma} space group. Upon finding this crystal disorder we became skeptical of the planar configuration reported for violuric acid monohydrate because of the similarities in structure between the two compounds. When investigating VAMH a double-welled potential and imaginary mode lead us to prematurely conclude that VAMH had a space group of \textit{P}2\textsubscript{1} rather than \textit{Cmc}2\textsubscript{1}. When evaluating the terahertz spectra of VAMH generated in the \textit{P}2\textsubscript{1} space group to the experimental there was not good peak alignment. This led us to increase the size of the unit cell and remove all symmetry operations, allowing the molecules to freely optimize. The molecules were then perceived to belong to the orthorhombic \textit{Pca}2\textsubscript{1} space group. In this space group the double-welled potential disappeared and the experimental spectra aligned well with the theoretical spectra. 
The second half of this dissertation focused on developing tools for structure determination via molecular dynamics. With a focus on organic macromolecules and polymers we developed a workflow that implemented Espaloma generated forcefields into the molecular dynamics framework. It was necessary to validate the accuracy of the Espaloma forcefields for conjugated, organic molecule before using the workflow to predict morphologies of these kinds of materials. To validate the workflow we predicted the morphologies of two widely studied organic molecules, perylene and poly-3-hexylthiophene (P3HT). We found that Espaloma produces similar forcefield parameters to the generalized amber forcefield (GAFF). The predicted morphologies of both materials with the espaloma forcefield showed great overlap with those predicted using the OPLS-UA forcefield. Once the workflow was validated for this use we utilized it to predict the morphologies of 13 conjugated donor-acceptor co-polymers over a range of temperatures. A function was written to calculate the persistence length from the predicted morphologies of the 13 polymers. This function was validated against persistence lengths calculated via DFT and small angle neutron scattering patterns. 